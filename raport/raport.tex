documentclass{article}
\usepackage{parskip}
\usepackage[utf8]{inputenc} 
\usepackage[T1]{fontenc} % Poprawne wyświetlanie polskich znaków
\usepackage[polish]{babel} % Polskie wsparcie językowe
\usepackage{datetime2} % Lepsze formatowanie daty

\begin{document}

\title{Analiza Szeregów Czasowych}
\author{
    Łukasz Jarząbkowski \and
    Mateusz Kuczyński \and
    Piotr Kowalczyk \and
    Wiktor Musiałowski \and
    Andrzej Bednarz
}
\date{\DTMdisplaydate{\year}{\month}{\day}{-1}} % Formatowanie daty

\maketitle

\section{Opis problemu}

W dzisiejszym dynamicznie rozwijającym się świecie, efektywne zarządzanie zużyciem energii jest kluczowym elementem zapewnienia stabilności i zrównoważonego rozwoju gospodarczego. W kontekście Stanów Zjednoczonych, jako jednego z największych konsumentów energii na świecie, analiza wzorców zużycia energii staje się szczególnie istotna. Projekt ten koncentruje się na analizie szeregów czasowych dziennych zużyć energii w Ameryce, mając na celu zrozumienie i prognozowanie zmian w zapotrzebowaniu na energię.

Głównym celem projektu jest zbadanie wzorców zużycia energii w USA na przestrzeni określonego okresu czasu. Dzięki analizie danych historycznych, możliwe będzie:

\begin{itemize}
    \item Identyfikacja sezonowych i długoterminowych trendów w zużyciu energii.
    \item Wykrycie ewentualnych anomalii oraz ich potencjalnych przyczyn.
    \item Prognozowanie przyszłego zużycia energii z zastosowaniem modeli statystycznych i uczenia maszynowego.
\end{itemize}

\section{Znaczenie Analizy Zużycia Energii}

Analiza zużycia energii jest niezbędna dla różnych interesariuszy, w tym:

\begin{itemize}
    \item Operatorów systemów energetycznych, którzy muszą zapewnić niezawodność dostaw energii.
    \item Decydentów politycznych, którzy opracowują strategie zrównoważonego rozwoju energetycznego.
    \item Przedsiębiorstw energetycznych, które planują inwestycje w infrastrukturę i zarządzanie zasobami.
    \item Konsumentów, którzy dążą do optymalizacji zużycia energii i redukcji kosztów.
\end{itemize}

\subsection{Metodologia}
% To w razie co może być do zmiany, jeszcze nie wiemy jakie modele konkretnie użyliśmy więc wpisałem tak se żeby było 
W projekcie zostaną wykorzystane zaawansowane metody analizy szeregów czasowych, takie jak:

\begin{itemize}
    \item Dekompozycja szeregów czasowych: analiza komponentów trendu, sezonowości i reszt.
    \item Modele autoregresyjne (AR) oraz modele zintegrowane autoregresyjne ze średnią ruchomą (ARIMA).
    \item Modele uczenia maszynowego: np. sieci neuronowe rekurencyjne (RNN) i ich zaawansowane wersje, takie jak LSTM (Long Short-Term Memory).
\end{itemize}

\subsection{Dane}

Dane użyte w projekcie pochodzą z platformy Kaggle, obejmują dzienne zużycie energii w różnych regionach USA. Zostaną one poddane wstępnej obróbce, w tym czyszczeniu, normalizacji i transformacji, aby zapewnić ich jakość i przydatność do analizy.

\subsection{Przewidywane Wyniki}

Projekt ma na celu dostarczenie:

\begin{itemize}
    \item Dokładnych modeli predykcyjnych, które mogą być użyte do planowania zużycia energii.
    \item Wglądów w czynniki wpływające na zmienność zużycia energii.
    \item Rekomendacji dla interesariuszy dotyczących zarządzania zużyciem energii.
\end{itemize}

\end{document}