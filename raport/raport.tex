\documentclass{article}
\usepackage{parskip}
\usepackage[utf8]{inputenc} 
\usepackage[T1]{fontenc} % Poprawne wyświetlanie polskich znaków
\usepackage[polish]{babel} % Polskie wsparcie językowe
\usepackage{datetime2} % Lepsze formatowanie daty
\usepackage{listings}
\usepackage{xcolor}
\usepackage{graphicx}

 \lstset{
  language=R,
  basicstyle=\ttfamily\small,
  keywordstyle=\color{blue},
  stringstyle=\color{red},
  commentstyle=\color{gray},
  showstringspaces=false,
  numbers=left,
  numberstyle=\tiny\color{gray},
  stepnumber=1,
  numbersep=5pt,
  breaklines=true,
  frame=single,
  captionpos=b,
}

\begin{document}
\title{Analiza Szeregów Czasowych}
\author{
    Łukasz Jarząbkowski \and
    Mateusz Kuczyński \and
    Piotr Kowalczyk \and
    Wiktor Musiałowski \and
    Andrzej Bednarz
}
\date{\DTMdisplaydate{\year}{\month}{\day}{-1}} % Formatowanie daty

\maketitle
\thispagestyle{empty} % Usunięcie numeracji strony na stronie tytułowej
\newpage 

\section{Opis problemu}

W dzisiejszym dynamicznie rozwijającym się świecie, efektywne zarządzanie zużyciem energii jest kluczowym elementem zapewnienia stabilności i zrównoważonego rozwoju gospodarczego. W kontekście Stanów Zjednoczonych, jako jednego z największych konsumentów energii na świecie, analiza wzorców zużycia energii staje się szczególnie istotna. Projekt ten koncentruje się na analizie szeregów czasowych dziennych zużyć energii w Ameryce, mając na celu zrozumienie i prognozowanie zmian w zapotrzebowaniu na energię.

Głównym celem projektu jest zbadanie wzorców zużycia energii w USA na przestrzeni określonego okresu czasu. Dzięki analizie danych historycznych, możliwe będzie:

\begin{itemize}
    \item Identyfikacja sezonowych i długoterminowych trendów w zużyciu energii.
    \item Wykrycie ewentualnych anomalii oraz ich potencjalnych przyczyn.
    \item Prognozowanie przyszłego zużycia energii z zastosowaniem modeli statystycznych i uczenia maszynowego.
\end{itemize}

\section{Znaczenie Analizy Zużycia Energii}

Analiza zużycia energii jest niezbędna dla różnych interesariuszy, w tym:

\begin{itemize}
    \item Operatorów systemów energetycznych, którzy muszą zapewnić niezawodność dostaw energii.
    \item Decydentów politycznych, którzy opracowują strategie zrównoważonego rozwoju energetycznego.
    \item Przedsiębiorstw energetycznych, które planują inwestycje w infrastrukturę i zarządzanie zasobami.
    \item Konsumentów, którzy dążą do optymalizacji zużycia energii i redukcji kosztów.
\end{itemize}

\subsection{Metodologia}

W projekcie zostaną wykorzystane zaawansowane metody analizy szeregów czasowych, takie jak:

\begin{itemize}
    \item Dekompozycja szeregów czasowych: analiza komponentów trendu i sezonowości.
    \item Modele autoregresyjne (AR) oraz modele zintegrowane autoregresyjne ze średnią ruchomą (ARIMA)
\end{itemize}

\subsection{Dane}

Dane użyte w projekcie pochodzą z platformy Kaggle, obejmują dzienne zużycie energii w różnych regionach USA. Zostaną one poddane wstępnej obróbce, w tym czyszczeniu, normalizacji i transformacji, aby zapewnić ich jakość i przydatność do analizy.

\subsection{Przewidywane Wyniki}

Projekt ma na celu dostarczenie:

\begin{itemize}
    \item Dokładnych modeli predykcyjnych, które mogą być użyte do planowania zużycia energii.
    \item Wglądów w czynniki wpływające na zmienność zużycia energii.
    \item Rekomendacji dla interesariuszy dotyczących zarządzania zużyciem energii.
\end{itemize}

\section{Opis rozwiązania problemu}
\subsection{Problem sezonowości danych}
 Zużycie energii często zmienia się w zależności od pory roku. W zimie zużycie energii na ogrzewanie wzrasta, natomiast w lecie może wzrosnąć zużycie energii na klimatyzację. Taka zmienność może utrudniać identyfikację długoterminowych trendów, ponadto występują również okresy świąt czy wakacje które równeż mogą powodowac nagłe zmiany w zużyciu energii które nie są reprezentatywne dla reszty sezonu.
 Kolejnym czynnikiem utrudniającym analize są anomalie pogodowe takie jak fale upałów czy mrozy, które mogą znacząco wpływać na zużycie energii i wprowadzać anomalie do danych.
 
 Ceny energii również mogą się zmieniać w zależności od sezonu, co może wpływać na zachowania konsumpcyjne. Na przykład, wzrost cen energii w zimie może skłaniać użytkowników do oszczędności.
 Postęp technologiczny  np. w dziedzinie efektywności energetycznej to ostatni z czynników który może wpływać na długoterminowe zużycie energii, co w konsekwencji może być trudne do oddzielenia od sezonowych wzorców.
 \newpage
 
 \subsection{Wstępna analiza danych i wykresy}
Do analizy danych wykorzystany został największy zestaw danych: PJME Hourly.
 Fragment kodu poniżej, zastępuje braki w danych bazowych za pomocą interpolacji liniowej a duplikaty przy pomocy średniej z duplikatów. 
\begin{lstlisting}[caption={Wczytanie plików z folderu dane i wstępna obróbka}]
raw_data <- read.csv("Dane/PJME_hourly.csv")
ts_data <- raw_data %>%
  rename(datetime = Datetime) %>%
  mutate(datetime = as_datetime(datetime)) %>%
  arrange(datetime)

ts_data %>% duplicates(index = datetime)

ts_data <- raw_data %>%
  rename(datetime = Datetime,
         h_energy_consumption = PJME_MW) %>%
  mutate(datetime = as_datetime(datetime)) %>%
  arrange(datetime) %>%
  group_by(datetime) %>% 
  summarise(h_energy_consumption = mean(h_energy_consumption, na.rm = TRUE)) %>%
  as_tsibble(index = datetime) %>%
  fill_gaps() %>% 
  mutate(h_energy_consumption = na_interpolation(h_energy_consumption))
\end{lstlisting}
 \newpage
\begin{lstlisting}[caption={Kod wywołujący wykres ogólny dla konsumpcji energii }]
autoplot(ts_data, h_energy_consumption) +
  labs(title = "Hourly Energy Consumption",
       subtitle = "PJM East Region: 2001-2018 (PJME)",
       x = "Datetime",
       y = "Energy Consumption (MW)") +
  theme_minimal() +
  theme(
    plot.title = element_text(size = 20, face = "bold", hjust = 0.5),
    plot.subtitle = element_text(size = 16, hjust = 0.5),
    axis.title.x = element_text(size = 14),
    axis.title.y = element_text(size = 14),
    axis.text = element_text(size = 12),
    plot.caption = element_text(size = 10),
    panel.grid.major = element_line(color = "grey80"),
    panel.grid.minor = element_line(color = "grey90"),
    plot.background = element_rect(fill = "white"),
    panel.background = element_rect(fill = "white")
  ) +
  scale_x_datetime(date_labels = "%Y-%m-%d", date_breaks = "5 years") +
  scale_y_continuous(labels = scales::comma)
\end{lstlisting}


\begin{figure}
    \includegraphics[width=1\linewidth]{image.png}
    \caption{Godzinne zużycie energii w latach 2001-2018}
    \label{}
\end{figure}

\newpage

Wykresy sezonowości rocznej, tygodniowej i dziennej przedstawiają się następująco:

\begin{figure}[h]
    \includegraphics[width=0.8\linewidth]{image2.png}
    \caption{Sezonowość roczna}
    \label{fig:seasonality-yearly}
\end{figure}

\begin{figure}[h]
    \includegraphics[width=0.8\linewidth]{image3.png}
    \caption{Sezonowość tygodniowa}
    \label{fig:seasonality-weekly}
\end{figure}

\begin{figure}[h]
    \includegraphics[width=1\linewidth]{image 4.png}
    \caption{sezonowość dzienna}
    \label{fig:enter-label}
\end{figure}

\newpage
\subsection{Modelowanie danych i prognoza}
  Dane dotyczące godzinowego zużycia energii mogą wykazywać regularne wzorce sezonowe w skali doby oraz do roku, ze względu na zmieniające się warunki oświetleniowe, pogodowe i klimatyczne. Model dynamicznej regresji harmonicznej jest odpowiedni do modelowania tego rodzaju sezonowości, ponieważ może uwzględnić zarówno składowe krótkookresowe (dzienną) jak i długookresowe Zakres dat danych obejmuje lata od 2001 do 2018 , co może oznaczać, że występują zmiany trendów oraz sezonowości w czasie. przy pomocy tego modelu można tę zmienność w czasie poprzez dostosowanie parametrów harmonicznych (częstotliwości i amplitudy) oraz innych zmiennych egzogenicznych, które mogą wpływać na zużycie energii.
    Dla porównania, wykorzystany został również model ARIMA aby porównać uzyskane wyniki, podstawą do modelowania jest sprawdzenie stacjonarności szeregu.

    
Z wykonanego testu adf wynika, że szereg czasowy jest stacjonarny a więc możliwe jest dalsze modelowanie danych za pomocą modelu regresji harmonicznej oraz Arima.
\begin{lstlisting}[caption={Test ADF }]
adf.test(as.ts(ts_data))
\end{lstlisting}
Dickey-Fuller = -24.094, Lag order = 52, p-value = 0.01
alternative hypothesis: stationary
\end{document}
